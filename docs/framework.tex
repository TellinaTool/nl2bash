%!TEX root=writeup.tex
\section{Approach}
\label{sec:framework}

\begin{figure}
    \begin{center} \includegraphics[width=4in]{architecture.pdf} \end{center}
    \caption{The overall architecture of our framework.}
    \label{fig:arch}
\end{figure}

\autoref{fig:arch} shows the major components of our translation framework.
Offline, the tool learns the syntax of common Bash commands using manpages. Both
the manpages and input-output examples obtained from StackOverflow are used to
learn an appropriate scoring function. At runtime, the tool enumerates possible
candidate programs using the learned syntax, scores them using the learned
scoring function, and presents the top-ranked suggestions to the user.

Section~\ref{subsec:represent} introduces the subset of Bash our tool supports
and how we obtain that subset automatically from manpages.
Section~\ref{subsec:parser} describes training and running the semantic parser.

\subsection{Language Design}
\label{subsec:represent}
An ideal intermediate language for the CLI programs should be expressive enough so that it covers many interesting real world cases, and concise enough for the synthesizer to efficiently synthesize. \autoref{fig:lang} presents the syntax of the intermediate language.

\begin{figure}[ht]
\[
\begin{array}{rlll}
\multicolumn{3}{l}{\textbf{Program}}\\
\mathit{p} & :=  & \mathit{cmd}~\overline{\mathit{option}} & \textrm{(Program)}\\
    &  & \mathit{p} ~\|~ \mathsf{xargs} ~\mathit{p} & \textrm{(Pipelined programs)} \\
\mathit{option} & := & \mathit{flag} & \textrm{(Command flag)}\\
                &    & \mathit{val} & \textrm{(Value)}\\
\mathit{cmd} & := & ... & \textrm{(Command names)}\\
\mathit{flag} & := & ... & \textrm{(Flag names)}\\
\mathit{val} & := & ...  & \textrm{(Primitive values)}\\
~\\
\multicolumn{3}{l}{\textbf{Command Signature}}\\
\mathit{sig} & := & \mathit{name}~\mathit{op} & \textrm{(Command signature)}\\
\mathit{op} &:= & \mathit{flag} & \textrm{(Flag)}\\
                &   & \mathit{arg}[\tau] & \textrm{(Argument)}\\
                &   & \mathit{arg}[\tau] ...& \textrm{(Argument list)}\\
                &   & \mathit{op}~|~\mathit{op} & \textrm{(Exclusive options)}\\
                &   & [ \mathit{op} ] & \textrm{(Optional options)}\\
~\\
\multicolumn{3}{l}{\textbf{Rules}}\\
\mathit{rule}^{*} &:=& \mathit{cmd}~\overline{\mathit{option}} : \tau & \textrm{(Command typing rule)}\\
~\\
\multicolumn{3}{l}{\textbf{Types}}\\
\tau_0 & := & \mathsf{void} ~|~ \mathsf{File} ~|~ \mathsf{Date} ~|~ ...& \textrm{(Primitive types)}\\
\tau & := & \tau_0 ~|~ (\bar{\tau}_0)\rightarrow \tau_0 & \textrm{(Type)}
\end{array}
\]
\caption{Intermediate Language syntax, where $\mathit{cmd}$ ranges over strings presernting command names, $\mathit{flag}$ are strings presenting flag names, $\mathit{val}$ presents values in bash programs, and $\mathit{arg}$ represents meta-argument names.}
\label{fig:lang}
\end{figure}


\textbf{Program} defines the formation rules for a program: i.e. a CLI program is either a primitive command or the pipeline of two programs. However, not all of programs that can be represented by the syntax is valid, e.g. the program \code{find -goal} is well formed under the syntax, but it is not a valid real CLI program, as the \code{goal} flag is not part of the command \code{find}. Thus, command signature is introduced to check the well-formedness of a given CLI program. 

The command signature defines the usage of a certain command: a CLI program is well formed only if it follows its corresponding structure. Besides, types help restrict usage of values in a program as other typed languages. Figure~\ref{fig:cmdsig} shows the signature of some commands used in our system. With this signature, a CLI program \code{mv -f -v a.txt b.txt} is will formed as it is captured by the signature of \code{mv}.

\begin{figure}[ht]
\[
\begin{array}{l}

\code{mv [-f | -i | -n] [-v] source:File target:File}\\
\code{mv [-f | -i | -n] [-v] source:File ... directory:File}\\
\code{sort [-bdfgiMnrckmosStTuz] [file:File ...]}\\
\code{grep [-abcdDEFGHhIiJLlmnOopqRSsUVvwxZ] [-A num:Number]} (... rest~omitted)\\
\code{cp [-R [-H | -L | -P]] [-fi | -n] [-apvX] source\_file:File target\_file:File}\\
\code{cp [-R [-H | -L | -P]] [-fi | -n] [-apvX] source\_file:File ... target\_directory:File}\\
\code{ls [-ABCFGHLOPRSTUW\@abcdefghiklmnopqrstuwx1] [file:File ...]}\\
...
\end{array}
\]
\vspace{-15pt}
\caption{Examples of some command signatures.}
\label{fig:cmdsig}
\end{figure}

Since the number of basic Linux commands and options is large even in domain-specific scenarios, it is difficult for a human designer to hand-code all of the generation rules. Though in the course project submission, we create the command signatures by hand, we plan as a longer-term project to semi-automate this step by adding information extracted from the Linux man pages\footnote{\url{http://linux.die.net/man/}}. With this support, extending our tool to handle new commands simply requires adding a new man page to the training data.
\subsection{Natural Language to Bash Command Translation}
\label{subsec:parser}

We adopted a top-down approach that searches for the most relevant bash commands for a given natural language description. Specifically, we designed an efficient greedy search procedure (\S~\ref{subsec:decoding}) that enumerates the set of relevant bash commands generated by the DSL grammar developed in the previous section, and extracts a set of features for each of them paired with the natural language description (\S~\ref{subsec:feature}). The set of candidate programs are then ranked by a linear scoring function, the weights of which are learned from a training set (\S~\ref{subsec:training}). 

\subsubsection{Decoding}
\label{subsec:decoding}

To control scalability, we consider only the following small set of head commands: \texttt{find, mv, sort, grep, cp, ls, tar, xargs}. Nevertheless, the number of possible bash commands covered by this set is still daunting. Take the \texttt{find} command as an example, it has nearly 40 options and hence $2^{40}$ bash command candidates\footnote{Most of the command-option combinations are pretty rare. However, we haven't found an effective way to guide search with this information}. To prevent the search space from blowing up, we consider only candidate programs that contain $2\sim 5$ terms (a ``term'' is either a head command or an option). We also employ a greedy search procedure which only explores the top-$k$ best children of each node based on the scoring of the partial command ending at the child node. We designed the features to be factorized over each term of the program, i.e. a set of features are extracted for each term in the command candidate and the final feature set is the union of them. This allows us features of bigger structures to be computed based on the features of its substructures. While this simplified feature set misses most information regarding both the structure of the program and the structure of the language, we found it to frequently capture the gist of a translation in practice. 

The greedy procedure for bash command enumeration and feature extraction is summarized in alg.~\ref{alg:decoding}.

\begin{algorithm} [ht]
\caption{Bash command enumeration and feature extraction\label{alg:decoding}}
\SetKwInOut{Input}{Input}
\SetKwInOut{Output}{Output}
\SetKwInOut{Initial}{Initialization}
\SetKwFunction{Add}{AddTo}
\SetKwFunction{DFS}{DFS}
\SetKwFunction{ExtractFeat}{ExtractFeatures}
\SetKwFunction{TopKBest}{TopKBestNextTerms}
\SetKwFunction{EndOfCmd}{IsEndOfCmd}
% \SetKwFunction{AccFeat}{AccFeaturesWithBacktracking}
\SetKwFunction{PrintCmd}{GetCmdAndFeaturesWithBacktracking}

\Input{nl\_cmd, HeadCmdSet}
\Output{CmdsAndFeatures}
\Initial{CmdsAndFeatures=\{\}}
\For {head\_cmd $\in$ HeadCmdSet}{
	\DFS{head\_cmd, nl\_cmd}
}
\SetKwProg{fun}{Procedure}{}{}
\fun{\DFS{term, nl\_cmd}}{	
	\ExtractFeat{term, nl\_cmd} \\
	\If {\EndOfCmd{term}} {
		CmdsAndFeatures $\leftarrow$ CmdsAndFeatures + \\
			\PrintCmd{term}
	}
	\For {next\_term $\in$ \TopKBest{term}} {
		next\_term.backpointer = term \\
		\DFS{next\_term, nl\_cmd}
	}
}
\fun{\ExtractFeat{term, nl\_cmd}}{
	FeatureSet = $\emptyset$
	
	\For {word $\in$ nl\_cmd} {
		FeatureSet = FeatureSet + (term, word)	
	}
	\KwRet{FeatureSet}
}
\end{algorithm}

\subsubsection{Features}
\label{subsec:feature}
The following features are used:
\begin{itemize}\itemsep-1pt
	\item association of key words/phrases to partial expressions
	\item association between partial expressions (e.g. how often do they combined in valid commands)
	\item similarity of key words/phrases in the command to the man page explanation text of a partial expression
	\item complexity of the logical formulas and the commands generated from them.
\end{itemize}
We use the structured perception algorithm to learn weights of the scoring function from the example pairs. In the training process, the input are pairs collected from StackOverflow, and the target is to learn a scoring functions to evaluate the correspondence between the a natural language sentence and a CLI program. In each training iteration, top ranked logical form are selected and the weights are updated based on its similarity to the ground truth logical form. 

\subsubsection{Training}
\label{subsec:training}

\begin{algorithm} 
\label{alg:training}
\caption{Perceptron Training}
\end{algorithm}
