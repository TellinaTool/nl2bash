% This is "sig-alternate.tex" V2.1 April 2013
% This file should be compiled with V2.5 of "sig-alternate.cls" May 2012
%
% This example file demonstrates the use of the 'sig-alternate.cls'
% V2.5 LaTeX2e document class file. It is for those submitting
% articles to ACM Conference Proceedings WHO DO NOT WISH TO
% STRICTLY ADHERE TO THE SIGS (PUBS-BOARD-ENDORSED) STYLE.
% The 'sig-alternate.cls' file will produce a similar-looking,
% albeit, 'tighter' paper resulting in, invariably, fewer pages.
%
% ----------------------------------------------------------------------------------------------------------------
% This .tex file (and associated .cls V2.5) produces:
%       1) The Permission Statement
%       2) The Conference (location) Info information
%       3) The Copyright Line with ACM data
%       4) NO page numbers
%
% as against the acm_proc_article-sp.cls file which
% DOES NOT produce 1) thru' 3) above.
%
% Using 'sig-alternate.cls' you have control, however, from within
% the source .tex file, over both the CopyrightYear
% (defaulted to 200X) and the ACM Copyright Data
% (defaulted to X-XXXXX-XX-X/XX/XX).
% e.g.
% \CopyrightYear{2007} will cause 2007 to appear in the copyright line.
% \crdata{0-12345-67-8/90/12} will cause 0-12345-67-8/90/12 to appear in the copyright line.
%
% ---------------------------------------------------------------------------------------------------------------
% This .tex source is an example which *does* use
% the .bib file (from which the .bbl file % is produced).
% REMEMBER HOWEVER: After having produced the .bbl file,
% and prior to final submission, you *NEED* to 'insert'
% your .bbl file into your source .tex file so as to provide
% ONE 'self-contained' source file.
%
% ================= IF YOU HAVE QUESTIONS =======================
% Questions regarding the SIGS styles, SIGS policies and
% procedures, Conferences etc. should be sent to
% Adrienne Griscti (griscti@acm.org)
%
% Technical questions _only_ to
% Gerald Murray (murray@hq.acm.org)
% ===============================================================
%
% For tracking purposes - this is V2.0 - May 2012

\documentclass{sig-alternate-05-2015}

\newcommand{\ktnote}[1]{\textcolor{blue}{#1}}
\newcommand{\eat}[1]{\ignorespaces}

\newcommand{\R}{\mathbb{R}}

\newcommand{\zeromatrix}{\mathbf{0}}
\newcommand{\bD}{\mathbf{D}}
\newcommand{\bE}{\mathbf{E}}
\newcommand{\bP}{\mathbf{P}}
\newcommand{\bX}{\mathbf{X}}
\newcommand{\br}{\mathbf{r}}
\newcommand{\bu}{\mathbf{u}}
\newcommand{\bw}{\mathbf{w}}
\newcommand{\bx}{\mathbf{x}}
\newcommand{\by}{\mathbf{y}}

\newcommand{\mC}{\mathcal{C}}
\newcommand{\mD}{\mathcal{D}}
\newcommand{\mE}{\mathcal{E}}
\newcommand{\mG}{\mathcal{G}}
\newcommand{\mH}{\mathcal{H}}
\newcommand{\mK}{\mathcal{K}}
\newcommand{\mL}{\mathcal{L}}
\newcommand{\mN}{\mathcal{N}}
\newcommand{\mO}{\mathcal{O}}
\newcommand{\mP}{\mathcal{P}}
\newcommand{\mR}{\mathcal{R}}

\newcommand{\iL}{\mathit{L}}
\newcommand{\iQ}{\mathit{Q}}
\newcommand{\iR}{\mathit{R}}
\newcommand{\iU}{\mathit{U}}
\newcommand{\iV}{\mathit{V}}
\newcommand{\iX}{\mathit{X}}
\newcommand{\iY}{\mathit{Y}}

\newcommand{\sigmoid}{\text{sigm}}
%\newcommand{\vector}{\text{vec}}
\newcommand{\KL}{\text{KL}}
\newcommand{\Loss}{\mathcal{L}}
\newcommand{\st}{\text{ s.t. }}

% Objective functions
\newcommand{\bilinear}[3]{#1^{\top}#2#3}
\newcommand{\bidiag}[3]{(#1 \circ #3) ^{\top} #2}
\newcommand{\diag}[1]{\text{diag}\left(#1\right)}
\newcommand{\biband}[3]{#1^{\top}#2#3}

% Dependency paths
\newcommand{\leftarc}[1]{$\leftarrow${[#1]}-}
\newcommand{\rightarc}[1]{-{[#1]}$\rightarrow$}
\newcommand{\wildcard}{$\langle*\rangle$}

\newcommand{\gene}[1]{{\texttt{#1}}}
\newcommand{\rev}[1]{#1$^{-1}$}

\newcommand{\neigh}[1]{\text{nb}\left(#1\right)}
\newcommand{\diagm}[1]{\text{diag}\left(#1\right)}
\newcommand{\ind}[1]{\mathbf{1}\left(#1\right)}
\newcommand{\argmin}[1]{\underset{#1}{\text{argmin}}}
%\newcommand{\vector}[1]{\underset{#1}{\text{vec}}}
\newcommand{\BigO}[1]{\ensuremath{\operatorname{O}\bigl(#1\bigr)}}
\newcommand{\cheading}[2]{\textbf{#1\hfill #2}}
\newcommand{\highest}[1]{\textbf{#1}}% == highest score

\newcommand{\cut}[1]{}
\newcommand{\victoria}[1]{\noindent{\textcolor{red}{\{{\bf victoria:} \em #1\}}}}

\begin{document}

% Copyright
\setcopyright{acmcopyright}
%\setcopyright{acmlicensed}
%\setcopyright{rightsretained}
%\setcopyright{usgov}
%\setcopyright{usgovmixed}
%\setcopyright{cagov}
%\setcopyright{cagovmixed}


% DOI
\doi{10.475/123_4}

% ISBN
\isbn{123-4567-24-567/08/06}

%Conference
\conferenceinfo{PLDI '13}{June 16--19, 2013, Seattle, WA, USA}

\acmPrice{\$15.00}

%
% --- Author Metadata here ---
% \conferenceinfo{WOODSTOCK}{'97 El Paso, Texas USA}
%\CopyrightYear{2007} % Allows default copyright year (20XX) to be over-ridden - IF NEED BE.
%\crdata{0-12345-67-8/90/01}  % Allows default copyright data (0-89791-88-6/97/05) to be over-ridden - IF NEED BE.
% --- End of Author Metadata ---

\title{Incremental Bash Synthesis from Natural Language\cut{\titlenote{(Produces the permission block, and
copyright information). For use with
SIG-ALTERNATE.CLS. Supported by ACM.}}}
\cut{\subtitle{[Extended Abstract]
\titlenote{A full version of this paper is available as
\textit{Author's Guide to Preparing ACM SIG Proceedings Using
\LaTeX$2_\epsilon$\ and BibTeX} at
\texttt{www.acm.org/eaddress.htm}}}}
%
% You need the command \numberofauthors to handle the 'placement
% and alignment' of the authors beneath the title.
%
% For aesthetic reasons, we recommend 'three authors at a time'
% i.e. three 'name/affiliation blocks' be placed beneath the title.
%
% NOTE: You are NOT restricted in how many 'rows' of
% "name/affiliations" may appear. We just ask that you restrict
% the number of 'columns' to three.
%
% Because of the available 'opening page real-estate'
% we ask you to refrain from putting more than six authors
% (two rows with three columns) beneath the article title.
% More than six makes the first-page appear very cluttered indeed.
%
% Use the \alignauthor commands to handle the names
% and affiliations for an 'aesthetic maximum' of six authors.
% Add names, affiliations, addresses for
% the seventh etc. author(s) as the argument for the
% \additionalauthors command.
% These 'additional authors' will be output/set for you
% without further effort on your part as the last section in
% the body of your article BEFORE References or any Appendices.

\numberofauthors{1} %  in this sample file, there are a *total*
% of EIGHT authors. SIX appear on the 'first-page' (for formatting
% reasons) and the remaining two appear in the \additionalauthors section.
%
\author{
% You can go ahead and credit any number of authors here,
% e.g. one 'row of three' or two rows (consisting of one row of three
% and a second row of one, two or three).
%
% The command \alignauthor (no curly braces needed) should
% precede each author name, affiliation/snail-mail address and
% e-mail address. Additionally, tag each line of
% affiliation/address with \affaddr, and tag the
% e-mail address with \email.
%
% 1st. author
\alignauthor
Xi Victoria Lin\\
       \affaddr{Computer Science \& Engineering}\\
       \affaddr{University of Washington}\\
       \affaddr{Seattle, WA 98195}\\
       \email{xilin@cs.washington.edu}
\cut{ % 2nd. author
\alignauthor
G.K.M. Tobin\titlenote{The secretary disavows
any knowledge of this author's actions.}\\
       \affaddr{Institute for Clarity in Documentation}\\
       \affaddr{P.O. Box 1212}\\
       \affaddr{Dublin, Ohio 43017-6221}\\
       \email{webmaster@marysville-ohio.com}
% 3rd. author
\alignauthor Lars Th{\o}rv{\"a}ld\titlenote{This author is the
one who did all the really hard work.}\\
       \affaddr{The Th{\o}rv{\"a}ld Group}\\
       \affaddr{1 Th{\o}rv{\"a}ld Circle}\\
       \affaddr{Hekla, Iceland}\\
       \email{larst@affiliation.org}
\and  % use '\and' if you need 'another row' of author names
% 4th. author
\alignauthor Lawrence P. Leipuner\\
       \affaddr{Brookhaven Laboratories}\\
       \affaddr{Brookhaven National Lab}\\
       \affaddr{P.O. Box 5000}\\
       \email{lleipuner@researchlabs.org}
% 5th. author
\alignauthor Sean Fogarty\\
       \affaddr{NASA Ames Research Center}\\
       \affaddr{Moffett Field}\\
       \affaddr{California 94035}\\
       \email{fogartys@amesres.org}
% 6th. author
\alignauthor Charles Palmer\\
       \affaddr{Palmer Research Laboratories}\\
       \affaddr{8600 Datapoint Drive}\\
       \affaddr{San Antonio, Texas 78229}\\
       \email{cpalmer@prl.com}
}
}
% There's nothing stopping you putting the seventh, eighth, etc.
% author on the opening page (as the 'third row') but we ask,
% for aesthetic reasons that you place these 'additional authors'
% in the \additional authors block, viz.
\additionalauthors{Additional authors: John Smith (The Th{\o}rv{\"a}ld Group,
email: {\texttt{jsmith@affiliation.org}}) and Julius P.~Kumquat
(The Kumquat Consortium, email: {\texttt{jpkumquat@consortium.net}}).}
\date{30 July 1999}
% Just remember to make sure that the TOTAL number of authors
% is the number that will appear on the first page PLUS the
% number that will appear in the \additionalauthors section.

\maketitle
\begin{abstract}
\end{abstract}


%
% The code below should be generated by the tool at
% http://dl.acm.org/ccs.cfm
% Please copy and paste the code instead of the example below. 
%
\begin{CCSXML}
<ccs2012>
<concept>
<concept_id>10011007.10011006.10011050.10011023</concept_id>
<concept_desc>Software and its engineering~Specialized application languages</concept_desc>
<concept_significance>300</concept_significance>
</concept>
</ccs2012>
\end{CCSXML}

\ccsdesc[300]{Software and its engineering~Specialized application languages}


%
% End generated code
%

%
%  Use this command to print the description
%
\printccsdesc

% We no longer use \terms command
%\terms{Theory}

\keywords{bash programming, program synthesis, natural language processing, crowdsourcing}

\section{Problem and Motivation}

The UNIX shell acts as an interface between the user and the kernel. It provides an environment that takes user commands and performs a variety of different tasks, including file system operations, gathering/summarizing information, network configuration, and more. Shell scripting is useful to automate repetitive tasks and can be useful to a wide range of users, for even some daily tasks take quite a bit longer to be done in a GUI than in the shell, for instance, {\it looking for files that hasn't been accessed for a certain amount of time}.

Shell commands are complex and non-uniform, 
%\footnote{The number of commands available on a Linux system differs drastically depending on the distro and packages installed. Commonly used commands s.a. {\tt find} can take more than 40 options, which is difficult to learn and to memorize.}
and even experienced IT professionals struggle to memorize every option necessary to carry out their intent. Previous study have shown that end-users can benefit from programming environments that are more natural and closer to the way they think about the task~\cite{Myers:2004:NPL:1015864.1015888}. We propose to let users express their intent in English, and to automatically translate them to shell commands.
% In an ideal scenario, the user could ``speak" to the kernel the same way he/she is speaking to another human being. In fact, many commonly performed system operations can be concisely described in short English sentences. 
Table~\ref{table:examples} shows a few of such examples.
 
Following recent work in programming by natural language (PBNL)~\cite{DBLP:conf/mobisys/LeGS13,DBLP:journals/corr/DesaiGHJKMRR15}, we train our model with shell commands paired with natural language descriptions using statistical machine learning methods. Previously, similar approaches have succeeded in translating natural language into database queries~\cite{DBLP:journals/pvldb/LiJ14} and API calls (e.g. Microsoft Excel string manipulation and IFTTT recipes) ~\cite{DBLP:conf/sigmod/GulwaniM14,DBLP:conf/acl/QuirkMG15}. The UNIX shell is likely to be more challenging because it is an extremely rich set of API functions, and it permits command cascades of unlimited depth through the ``pipe" mechanism.
% On one hand, it is possible that data-driven PBNL techniques could also succeed in shell commands synthesis. On the other hand, training a shell command synthesizer would require much more data than previous applications, such as database interface~\cite{DBLP:journals/pvldb/LiJ14} and Microsoft Excel string manipulation~\cite{DBLP:conf/sigmod/GulwaniM14}. Besides, shell , which further complicates the search space.
% and there is great value in providing a natural language interface to functions that is inefficient to be carried out with GUI. Especially
\begin{center}
\begin{table}[t]
\begin{tabular}{p{1.5in}p{1.5in}}
    \textbf{Natural Language} & \textbf{Command} \\
    \hline \hline
    count how many times ``at" occurs in ``/etc/passwd" %
        % \footnote{\url{https://stackoverflow.com/questions/1358540/how-to-count-all-the-lines-of-code-in-a-directory-recursively}} 
        &
        \texttt{grep -c "at" /etc/passwd} \\
    \hline
     display the 5 largest files in the current directory and its subdirectory. %
         % \footnote{\url{http://pro-toolz.net/data/programming/bash/important_shell_commands.html}}
         &
         \texttt{find . -type f -exec ls -s {} \; | sort -n -r | head -5} \\
     \hline
     find files and directories modified in the last 7 days%
         % \footnote{\url{http://www.bashoneliners.com/oneliners/oneliner/80/}}
         &
         \texttt{find . -mtime -7} \\
\end{tabular}
\caption{Shell Commands and Their Natural Language Descriptions.}
\label{table:examples}
\end{table}
\end{center}


\section{Background and Related Work}
The idea of natural language based computer interface was born even before the UNIX sytem was released~\cite{sammet1966use,Ballard:1979:PNL:800177.810072}. The central idea was to build compilers that compile natural language into lower-level instructions. However, NLP techniques were preliminary back then, and those systems, relying on manually engineered templates and grammars, had failed to handle the rich semantics and ambiguity in natural languages. 
% Besides, some researchers pointed out that making machines to understand our native tongue is a fruitless trial for it lacks mathematical support and programming languages remains simple in the formal system. In short, it was easier for people to learn how to program than for machines to learn natural language. 

While it may be unrealistic (and also fruitless~\cite{Dijkstra:1978:FNL:647639.760596}) to simply adopt natural language as a programming language, recent work have shown promise on synthesizing high-level code snippets from natural language descriptions~\cite{gulwani2010dimensions,DBLP:journals/corr/DesaiGHJKMRR15}.  
% and can be furthur improved through interaction with users~\cite{DBLP:conf/sigmod/GulwaniM14,DBLP:journals/pvldb/LiJ14}.
Despite the initial success of natural language interface to database~\cite{Popescu:2003:TTN:604045.604070,DBLP:journals/pvldb/LiJ14} and commercial applications such as the Microsoft Excel~\cite{DBLP:conf/mobisys/LeGS13,DBLP:conf/acl/QuirkMG15}, previous work on shell programming from natural language have shown that the task is extremely difficult~\cite{bashsynthesis,cozzie2011macho,cozzie2012macho,Pedersen-Report}. 
% In contrast to the previously introduced programming by natural language techniques, 
Macho~\cite{cozzie2011macho} is a system that can extract specifications from natural language text in the Linux man-pages\footnote{https://www.kernel.org/doc/man-pages/}, and generates candidate programs based on that. However, their approach only worked on generating the head commands.~\cite{cozzie2012macho} combines the natural language specifications with an example given by the user, and shows that the combination effectively prunes the search space, but the results were still limited to commands of length 2 or smaller.~\cite{bashsynthesis} attempted to synthesis file system operation commands using StackOverflow question-answer pairs, and have succeed in generating commands of length 5 or smaller. However, due to the lack of high-quality training data, big challenge still remains for the system to be practically useful.


\section{Approach and Uniqueness}

% In our work, three approaches are essential to build a natural language based shell command synthesizer that is practically useful. 
Our approach consists of three parts. First, we hire part-time programmers to collect shell commands and their natural language descriptions from the massive amount of shell-scripting tutorials and forums online, which results in a dataset that is orders of magnitude larger than those in previous works. Second, we restrict the domain to only commonly used file system operations to dramatically narrow the search space. We measure the ``commonness" based on the statistics of shell commands occurred in the crowdsourcing data. Third, we make use of the state-of-the-art incremental parsing techniques in NLP to increasing the search efficiency~\cite{zhao2014type,dyer2015transition,huang2010dynamic}, which allows us to explore a richer semantic space much faster. This framework can be generalized to more scripting languages.

\subsection{Havesting the Training Examples}

We built a web application that allows users to search\footnote{Powered by Google Search API: \url{https://cse.google.com/}.} for web pages that contains commonly used file system shell commands and their corresponding English descriptions. Examples of such web pages include tutorials, tech blogs, question-answering forums and course materials (*.pdf, *.doc, *.txt, etc.). The users are then asked to browse each retrieved web page and select all pairs of shell commands and their natural language descriptions found on the web page. We restrict the shell commands to be one-liners and the natural language description to be a single sentence. For the most commonly used commands, the natural language description can usually be copied verbatim from the web pages. The rest of the cases require the user to come up with their own descriptions using their background knowledge and the information on the web page. The users in our study are freelancers who are familiar with shell scripting. % hired using Upwork\footnote{\url{https://www.upwork.com/}}. 
The first two freelancers we hired collect 70 and 71 pairs respectively, over a 2-hour window. 

\subsection{Pruning the Search Space}

Since almost any operation performed by the operating system can be written in a shell script, it is important to select a subset of functionality to cover as a first step. 
In this work, we specifically look at file system operations. We restrict the first token of the command to be {\tt find}, but allow pipe structures that redirect the output of a {\tt find} command to other commands. At first glance, this seems to drastically cut the search space. However, the problem still remains difficult, for the user may describe the set of files/directories (s)he is looking for using \emph{name}, \emph{type}, \emph{permission}, \emph{size}, \emph{access time} and more. Extracting those attributes from the natural language is challenging. Besides, solving the ``find" problem is specifically valuable, for it gives us a filter on the files, and more functionality may be implemented by redirecting the filter output to other commands.
% since many requests can be accomplished using {\tt find} even if they are not explicitly searching for anything. Examples of such queries include \textit{sort files by size in the current directory} and \textit{check if my file has 644 permission}.   such as \textit{removing files older than 30 days},
% The ``commonness" is 
We hope that the training set we gathered automatically reflects the popularity of attributes users tend to specify, and hence the frequently occurred ones can be well learnt\footnote{It is possible that people tend to not post certain commands due to complexity and other reasons and we end up getting a biased training set. It this turned out to be the case, we will develop methods to compensate for the bias.}.

To make the trained model more generalizable, we normalize the command arguments by replacing the attribute values with their semantic types, 
% \footnote{Argument normalization may not be necessary if we decide to use neural network based models, for the models can automatically assign similar representations to arguments of the same type.}, 
as shown in figure~\ref{fig:norm}. The types are derived semi-automatically using man-pages.

\begin{figure}[t]
    \centering
    \includegraphics[width=3.3in]{normalization}
    \caption{Normalizing command arguments and the corresponding English noun phrases.}
    \label{fig:norm}
\end{figure}

\subsection{Incremental Parsing and Synthesis}

Most existing PBNL framework adopted a top-down approach that enumerates all legal programs according to the grammar of the programming language, and use the natural language to guide the enumeration~\cite{DBLP:journals/corr/DesaiGHJKMRR15,bashsynthesis}. For example,~\cite{bashsynthesis} applied top-down beam search with simple token based scoring. For head commands with a large set of options such as {\tt find}, enumerating all possible commands up to length 5 takes about 3s for a small beam size of 7, indicating the need for more aggressive pruning.

In this work, we plan to take a different approach that synthesizes the command bottom-up. The idea is based on the observation that the order of the shell command options is exchangable; therefore most of the time there exists a correct command whose options are in the same order as their corresponding natural language components\footnote{However, this assumption may be broken for commands with the ``pipe" structure. Noticed that in table~\ref{table:examples}, the first and the third command has options in the same order as their natural language correspondences. The second command exactly reversed the order in the natural language specification, and it contains one ``pipe"}. Inspired by existing work on shift-reduce natural language parsing~\cite{zhao2014type,dyer2015transition,huang2010dynamic}, we will train a synthesizer that parses the natural language incrementally, and maps the parsed natural language components into partial shell commands on the fly. % This process is illustrated in figure~\ref{}.
% \input{evaluation}
% \input{result}
\section{Contributions}





\cut{
We have already seen several typeface changes in this sample.  You
can indicate italicized words or phrases in your text with
the command \texttt{{\char'134}textit}; emboldening with the
command \texttt{{\char'134}textbf}
and typewriter-style (for instance, for computer code) with
\texttt{{\char'134}texttt}.  But remember, you do not
have to indicate typestyle changes when such changes are
part of the \textit{structural} elements of your
article; for instance, the heading of this subsection will
be in a sans serif\footnote{A third footnote, here.
Let's make this a rather short one to
see how it looks.} typeface, but that is handled by the
document class file. Take care with the use
of\footnote{A fourth, and last, footnote.}
the curly braces in typeface changes; they mark
the beginning and end of
the text that is to be in the different typeface.

You can use whatever symbols, accented characters, or
non-English characters you need anywhere in your document;
you can find a complete list of what is
available in the \textit{\LaTeX\
User's Guide}\cite{Lamport:LaTeX}.

\subsection{Math Equations}
You may want to display math equations in three distinct styles:
inline, numbered or non-numbered display.  Each of
the three are discussed in the next sections.

\subsubsection{Inline (In-text) Equations}
A formula that appears in the running text is called an
inline or in-text formula.  It is produced by the
\textbf{math} environment, which can be
invoked with the usual \texttt{{\char'134}begin. . .{\char'134}end}
construction or with the short form \texttt{\$. . .\$}. You
can use any of the symbols and structures,
from $\alpha$ to $\omega$, available in
\LaTeX\cite{Lamport:LaTeX}; this section will simply show a
few examples of in-text equations in context. Notice how
this equation: \begin{math}\lim_{n\rightarrow \infty}x=0\end{math},
set here in in-line math style, looks slightly different when
set in display style.  (See next section).

\subsubsection{Display Equations}
A numbered display equation -- one set off by vertical space
from the text and centered horizontally -- is produced
by the \textbf{equation} environment. An unnumbered display
equation is produced by the \textbf{displaymath} environment.

Again, in either environment, you can use any of the symbols
and structures available in \LaTeX; this section will just
give a couple of examples of display equations in context.
First, consider the equation, shown as an inline equation above:
\begin{equation}\lim_{n\rightarrow \infty}x=0\end{equation}
Notice how it is formatted somewhat differently in
the \textbf{displaymath}
environment.  Now, we'll enter an unnumbered equation:
\begin{displaymath}\sum_{i=0}^{\infty} x + 1\end{displaymath}
and follow it with another numbered equation:
\begin{equation}\sum_{i=0}^{\infty}x_i=\int_{0}^{\pi+2} f\end{equation}
just to demonstrate \LaTeX's able handling of numbering.

\subsection{Citations}
Citations to articles \cite{bowman:reasoning,
clark:pct, braams:babel, herlihy:methodology},
conference proceedings \cite{clark:pct} or
books \cite{salas:calculus, Lamport:LaTeX} listed
in the Bibliography section of your
article will occur throughout the text of your article.
You should use BibTeX to automatically produce this bibliography;
you simply need to insert one of several citation commands with
a key of the item cited in the proper location in
the \texttt{.tex} file \cite{Lamport:LaTeX}.
The key is a short reference you invent to uniquely
identify each work; in this sample document, the key is
the first author's surname and a
word from the title.  This identifying key is included
with each item in the \texttt{.bib} file for your article.

The details of the construction of the \texttt{.bib} file
are beyond the scope of this sample document, but more
information can be found in the \textit{Author's Guide},
and exhaustive details in the \textit{\LaTeX\ User's
Guide}\cite{Lamport:LaTeX}.

This article shows only the plainest form
of the citation command, using \texttt{{\char'134}cite}.
This is what is stipulated in the SIGS style specifications.
No other citation format is endorsed or supported.

\subsection{Tables}
Because tables cannot be split across pages, the best
placement for them is typically the top of the page
nearest their initial cite.  To
ensure this proper ``floating'' placement of tables, use the
environment \textbf{table} to enclose the table's contents and
the table caption.  The contents of the table itself must go
in the \textbf{tabular} environment, to
be aligned properly in rows and columns, with the desired
horizontal and vertical rules.  Again, detailed instructions
on \textbf{tabular} material
is found in the \textit{\LaTeX\ User's Guide}.

Immediately following this sentence is the point at which
Table 1 is included in the input file; compare the
placement of the table here with the table in the printed
dvi output of this document.

\begin{table}
\centering
\caption{Frequency of Special Characters}
\begin{tabular}{|c|c|l|} \hline
Non-English or Math&Frequency&Comments\\ \hline
\O & 1 in 1,000& For Swedish names\\ \hline
$\pi$ & 1 in 5& Common in math\\ \hline
\$ & 4 in 5 & Used in business\\ \hline
$\Psi^2_1$ & 1 in 40,000& Unexplained usage\\
\hline\end{tabular}
\end{table}

To set a wider table, which takes up the whole width of
the page's live area, use the environment
\textbf{table*} to enclose the table's contents and
the table caption.  As with a single-column table, this wide
table will ``float" to a location deemed more desirable.
Immediately following this sentence is the point at which
Table 2 is included in the input file; again, it is
instructive to compare the placement of the
table here with the table in the printed dvi
output of this document.


\begin{table*}
\centering
\caption{Some Typical Commands}
\begin{tabular}{|c|c|l|} \hline
Command&A Number&Comments\\ \hline
\texttt{{\char'134}alignauthor} & 100& Author alignment\\ \hline
\texttt{{\char'134}numberofauthors}& 200& Author enumeration\\ \hline
\texttt{{\char'134}table}& 300 & For tables\\ \hline
\texttt{{\char'134}table*}& 400& For wider tables\\ \hline\end{tabular}
\end{table*}
% end the environment with {table*}, NOTE not {table}!

\subsection{Figures}
Like tables, figures cannot be split across pages; the
best placement for them
is typically the top or the bottom of the page nearest
their initial cite.  To ensure this proper ``floating'' placement
of figures, use the environment
\textbf{figure} to enclose the figure and its caption.

This sample document contains examples of \textbf{.eps} files to be
displayable with \LaTeX.  If you work with pdf\LaTeX, use files in the
\textbf{.pdf} format.  Note that most modern \TeX\ system will convert
\textbf{.eps} to \textbf{.pdf} for you on the fly.  More details on
each of these is found in the \textit{Author's Guide}.

\begin{figure}
\centering
\includegraphics{fly}
\caption{A sample black and white graphic.}
\end{figure}

\begin{figure}
\centering
\includegraphics[height=1in, width=1in]{fly}
\caption{A sample black and white graphic
that has been resized with the \texttt{includegraphics} command.}
\end{figure}


As was the case with tables, you may want a figure
that spans two columns.  To do this, and still to
ensure proper ``floating'' placement of tables, use the environment
\textbf{figure*} to enclose the figure and its caption.
and don't forget to end the environment with
{figure*}, not {figure}!

\begin{figure*}
\centering
\includegraphics{flies}
\caption{A sample black and white graphic
that needs to span two columns of text.}
\end{figure*}


\begin{figure}
\centering
\includegraphics[height=1in, width=1in]{rosette}
\caption{A sample black and white graphic that has
been resized with the \texttt{includegraphics} command.}
\vskip -6pt
\end{figure}

\subsection{Theorem-like Constructs}
Other common constructs that may occur in your article are
the forms for logical constructs like theorems, axioms,
corollaries and proofs.  There are
two forms, one produced by the
command \texttt{{\char'134}newtheorem} and the
other by the command \texttt{{\char'134}newdef}; perhaps
the clearest and easiest way to distinguish them is
to compare the two in the output of this sample document:

This uses the \textbf{theorem} environment, created by
the\linebreak\texttt{{\char'134}newtheorem} command:
\newtheorem{theorem}{Theorem}
\begin{theorem}
Let $f$ be continuous on $[a,b]$.  If $G$ is
an antiderivative for $f$ on $[a,b]$, then
\begin{displaymath}\int^b_af(t)dt = G(b) - G(a).\end{displaymath}
\end{theorem}

The other uses the \textbf{definition} environment, created
by the \texttt{{\char'134}newdef} command:
\newdef{definition}{Definition}
\begin{definition}
If $z$ is irrational, then by $e^z$ we mean the
unique number which has
logarithm $z$: \begin{displaymath}{\log e^z = z}\end{displaymath}
\end{definition}

Two lists of constructs that use one of these
forms is given in the
\textit{Author's  Guidelines}.
 
There is one other similar construct environment, which is
already set up
for you; i.e. you must \textit{not} use
a \texttt{{\char'134}newdef} command to
create it: the \textbf{proof} environment.  Here
is a example of its use:
\begin{proof}
Suppose on the contrary there exists a real number $L$ such that
\begin{displaymath}
\lim_{x\rightarrow\infty} \frac{f(x)}{g(x)} = L.
\end{displaymath}
Then
\begin{displaymath}
l=\lim_{x\rightarrow c} f(x)
= \lim_{x\rightarrow c}
\left[ g{x} \cdot \frac{f(x)}{g(x)} \right ]
= \lim_{x\rightarrow c} g(x) \cdot \lim_{x\rightarrow c}
\frac{f(x)}{g(x)} = 0\cdot L = 0,
\end{displaymath}
which contradicts our assumption that $l\neq 0$.
\end{proof}

Complete rules about using these environments and using the
two different creation commands are in the
\textit{Author's Guide}; please consult it for more
detailed instructions.  If you need to use another construct,
not listed therein, which you want to have the same
formatting as the Theorem
or the Definition\cite{salas:calculus} shown above,
use the \texttt{{\char'134}newtheorem} or the
\texttt{{\char'134}newdef} command,
respectively, to create it.

\subsection*{A {\secit Caveat} for the \TeX\ Expert}
Because you have just been given permission to
use the \texttt{{\char'134}newdef} command to create a
new form, you might think you can
use \TeX's \texttt{{\char'134}def} to create a
new command: \textit{Please refrain from doing this!}
Remember that your \LaTeX\ source code is primarily intended
to create camera-ready copy, but may be converted
to other forms -- e.g. HTML. If you inadvertently omit
some or all of the \texttt{{\char'134}def}s recompilation will
be, to say the least, problematic.
}
%\end{document}  % This is where a 'short' article might terminate

%ACKNOWLEDGMENTS are optional
\cut{
\section{Acknowledgments}
This section is optional; it is a location for you
to acknowledge grants, funding, editing assistance and
what have you.  In the present case, for example, the
authors would like to thank Gerald Murray of ACM for
his help in codifying this \textit{Author's Guide}
and the \textbf{.cls} and \textbf{.tex} files that it describes.
}

%
% The following two commands are all you need in the
% initial runs of your .tex file to
% produce the bibliography for the citations in your paper.
\bibliographystyle{abbrv}
\bibliography{sigproc}  % sigproc.bib is the name of the Bibliography in this case
% You must have a proper ".bib" file
%  and remember to run:
% latex bibtex latex latex
% to resolve all references
%
% % ACM needs 'a single self-contained file'!
%
%APPENDICES are optional
%\balancecolumns
\appendix
%Appendix A
\section{Headings in Appendices}
The rules about hierarchical headings discussed above for
the body of the article are different in the appendices.
In the \textbf{appendix} environment, the command
\textbf{section} is used to
indicate the start of each Appendix, with alphabetic order
designation (i.e. the first is A, the second B, etc.) and
a title (if you include one).  So, if you need
hierarchical structure
\textit{within} an Appendix, start with \textbf{subsection} as the
highest level. Here is an outline of the body of this
document in Appendix-appropriate form:
\subsection{Introduction}
\subsection{The Body of the Paper}
\subsubsection{Type Changes and  Special Characters}
\subsubsection{Math Equations}
\paragraph{Inline (In-text) Equations}
\paragraph{Display Equations}
\subsubsection{Citations}
\subsubsection{Tables}
\subsubsection{Figures}
\subsubsection{Theorem-like Constructs}
\subsubsection*{A Caveat for the \TeX\ Expert}
\subsection{Conclusions}
\subsection{Acknowledgments}
\subsection{Additional Authors}
This section is inserted by \LaTeX; you do not insert it.
You just add the names and information in the
\texttt{{\char'134}additionalauthors} command at the start
of the document.
\subsection{References}
Generated by bibtex from your ~.bib file.  Run latex,
then bibtex, then latex twice (to resolve references)
to create the ~.bbl file.  Insert that ~.bbl file into
the .tex source file and comment out
the command \texttt{{\char'134}thebibliography}.
% This next section command marks the start of
% Appendix B, and does not continue the present hierarchy
\section{More Help for the Hardy}
The sig-alternate.cls file itself is chock-full of succinct
and helpful comments.  If you consider yourself a moderately
experienced to expert user of \LaTeX, you may find reading
it useful but please remember not to change it.
%\balancecolumns % GM June 2007
% That's all folks!
\end{document}